\startchapter{Experiments}
\label{chapter:Exp}

In this chapter, I try to answer the research question: Can the described P-Frame schema improve the compression of pieces of music with repeated instrumental tones without adding perceptible differences? My hypothesis is that by reusing portions of audio that are being repeated (P-Frames), as long as a piece of music has repeating structure, the scheme should be able to compress better as compared to the compression of every single frame individually (I-Frames). Therefore, to answer the research question quantitatively, it is required to have a statistical analysis of numerical outputs (Dependent Variables) of the scheme that is run over a random collection of pieces of music while controlling all different variables to isolate our conditions (Independent Variables). The following describes my experimental design:

\begin{description}
\item[Population:] To have an unbiased and random set of pieces of music, I used the "test" set in MUSDB-HQ which consists of 50 full-length songs ($\sim$3h duration) with different genres along with their isolated drums, bass, vocals and others stems \cite{musdb18-hq}.

\item[Dependent Variables (DVs):] After running the compression scheme on each audio file, I measure the following:
\begin{itemize}
\item\textbf{Audio Quality:} the audio quality was measured using Objective Difference Grade (ODG) of Perceptual Evaluation of Audio Quality (PEAQ) \cite{peaq} to ensure that the compression scheme does not add perceptible artifacts. The range is between 0 and -4.
\item\textbf{Compression Ratio:} As each song has a different length, the output size of files is compared by measuring the ratio, so the range is between 0 and 1. It is calculated as:  Compressed File Size / Uncompressed File Size
\item\textbf{Desirability:} As ODG and Compression-Ratio are correlated, I need a new variable that can help me interpret the effectiveness of the scheme. This measure is named Desirability, has a range between 0 and -1, and is calculated as: (ODG/4) / Compression Ratio
\item\textbf{P-Ratio:} To measure the number of frames that are reused in the compressed file. P-Ratio has a range between 0 and 1 and is calculated as: N(used P-Frames) / N(all frames)
% \item MAYBE NOT NEEDED: Reference Distance Histogram represented in 25\%, 50\%, and 75\% quantiles.
\end{itemize}

\item[Control Variables:] To ensure that we measure only the effectiveness of reused frames, all other techniques of audio compression should be disabled and non-zero parameters should be set to the lowest amount. Therefore, after running the MDCT algorithm, the following parameters are used:
\begin{itemize}
\item\textbf{High Frequency Cut} = 0
\item\textbf{Rounding Decimals} = 0
\item\textbf{P-Frame Zero Threshold} = ${10}^{-5}$; this value can not be zero. So because MDCT numbers are between -1 and 1 with 12 digits of decimal points, we use the mentioned small value.
\item\textbf{I-Frame Zero Threshold} = ${10}^{-5}$; to be equal to the other threshold.
\item\textbf{I-Rate} = 2 seconds; Only in section \ref{sec:pframe-exp}.
\end{itemize}

\item[Independent Variables (IVs):] The following variables are purposely set and the scheme is executed once per each of them. So each song is compressed 42 ($=6*3 + 6*4$) times.
\begin{itemize}[leftmargin=*]
\item\textbf{Audio Type} (6 levels)
\begin{enumerate}[nosep, leftmargin=*]
\item\textbf{Mixture:} The original song including vocals, drums, and the rest of the instruments.
\item\textbf{Vocals:} Only the vocal sound of the song which normally doesn't have repeating tones.
\item\textbf{Drums:} Only the drums sound of the song which normally repeats with the beat.
\item\textbf{REPET's Foreground:} The masked foreground of the REPET algorithm which is determined based on lack of repetition.
\item\textbf{REPET's Background:} The masked background of the REPET algorithm which is determined based on having repetitions.
\item\textbf{Sum:} A mixture of the latter two which holds the concatenation of REPET's foreground with only I-Frames and REPET's background with P-Frames.
\end{enumerate}
\item\textbf{Scheme} (3 levels); Only used in section \ref{sec:pframe-exp}.
\item\textbf{I-Rate} (4 levels); Only used in section \ref{sec:i-rates}.
\end{itemize}

\item[Statistical Analysis:] For each DV, I provide figures that show the mean of outputs in each one of the IVs ($N=50$). Error bars in graphs are all showing 95\% confidence intervals. In addition to the graphs, descriptive and inferential statistics are provided. Data is analyzed using a 2-way repeated-measures ANOVA and Tukey HSD post-hoc test. For every ANOVA test, the assumptions of Normality and Sphericity of Distributions were checked. Despite some mild violations of Normality, there was no violation of the Sphericity test. This means degrees of freedom never needed to be corrected. As for violations of the Normality test, it is believed that the ANOVA test is robust in groups of the same size.

\end{description}


\section{Effects of I-Rate on the Compression Scheme}
\label{sec:i-rates}

As described in section \ref{sec:idea}, the I-Rate parameter is the rate of forced I-Frames that P-Frames are not allowed to refer to any frame before them. A larger number in the I-Rate means P-Frames are allowed to look more back in time to find a suitable referenced frame to reuse. Additionally, it means more comparisons and more CPU time. In this section, I try to answer the question: What is the suitable amount of time for P-Frames to look back? My hypothesis is that, in the worst case, beats are repeated once per second. Therefore, 2 seconds for the I-Rate should be sufficient and there is no point in going back more in time to find a reference.

To test this hypothesis, DVs are measured as described at the beginning of chapter \ref{chapter:Exp} with an additional 4-level IV named I-Rates: 0.1 of a second, 1 second, 2 seconds, and 5 seconds. The following sections are reports for each measured variable.

\subsection{Audio Quality}

\begin{figure}[ht]
  \includesvg[inkscapelatex=false,  width=\linewidth]{Figures/chap5/irate-peaq.svg}
  \caption[Comparison of Audio Quality in different types of audio and I-Rates.]{Comparison of Audio Quality in different types of audio and I-Rates. Audio Quality is measured using Objective Difference Grade (ODG) in Perceptual Evaluation of Audio Quality (PEAQ) (0 = Imperceptible, -1 = Perceptible, but not annoying)}
  \label{fig:irate-peaq}
\end{figure}

\begin{table}[ht]
\fontsize{8}{10}\selectfont
\centering
\begin{tabularx}{\linewidth}{|m{3cm}||Y|Y||Y|Y||Y|Y||Y|Y|}
\hline
I-Rates & \multicolumn{2}{c||}{0.1} & \multicolumn{2}{c||}{1} & \multicolumn{2}{c||}{2} & \multicolumn{2}{c|}{5} \\
\hline
& Mean & SD & Mean & SD & Mean & SD & Mean & SD \\
\hline
Mixture & .15 & .037 & .125 & .099 & .114 & .137 & .105 & .166 \\
\hline
Vocals & .014 & .18 & -.349 & .513 & -.46 & .645 & -.603 & .805 \\
\hline
Drums & .051 & .187 & -.111 & .41 & -.161 & .479 & -.162 & .464 \\
\hline
REPET's Foreground & .092 & .064 & -.036 & .185 & -.065 & .194 & -.107 & .25 \\
\hline
REPET's Background & .143 & .039 & .112 & .098 & .098 & .165 & .089 & .187 \\
\hline
Sum & .148 & .038 & .121 & .107 & .111 & .148 & .104 & .167 \\
\hline
\end{tabularx}
\caption[Table of Audio Quality in different types of audio and I-Rates.]{Table of Audio Quality in different types of audio and I-Rates. Audio Quality is measured using Objective Difference Grade (ODG) in Perceptual Evaluation of Audio Quality (PEAQ) (0 = Imperceptible, -1 = Perceptible, but not annoying)}
\label{tab:irate-peaq}
\end{table}

It can be seen in figure \ref{fig:irate-peaq} and table \ref{tab:irate-peaq} that mixture, REPET's background, and the Sum audio inputs are imperceptible in all I-Rates.
In drums and REPET's foreground, however, with the increment of I-Rate, the audio quality is slightly reduced.
Most noticeably, with the increment of I-Rate, the audio quality in vocals are highly reduced.
Within-subject ANOVA showed that the audio quality is significantly affected by the used input audio type ($F(5,245)=31.4, p<.000, \eta_{p}^{2}=.39$), the I-Rate ($F(3,147)=44.8, p<.000, \eta_{p}^{2}=.477$), and the interaction between them ($F(15,735)=18.7, p<.000, \eta_{p}^{2}=.276$).
Tukey HSD post-hoc showed that the audio quality is significantly higher in I-Rate of 0.1 of a second ($M=.1, SD=.122$), as compared to other used I-Rates (all $p$s $<.001$) while it showed no significant reduction of audio quality for the I-Rate of 1 second ($M=-.023, SD=.333$), 2 seconds ($M=-.06, SD=.407$), and 5 seconds ($M=-.096, SD=.479$). In vocals specifically, Tukey HSD showed no significant reduction of audio quality between I-Rates of 1 second and 2 seconds, and I-Rates of 2 seconds and 5 seconds. However, the audio quality between I-Rates of 1 second and 5 seconds are significantly lowered ($p=.011$).

In summary, the results confirmed that, except for vocals, the lossy compression remains imperceptible with any amount of I-Rate and it stabilizes when it is higher than 1 second.

\subsection{Compression Ratio}

\begin{figure}[ht]
  \includesvg[inkscapelatex=false,  width=\linewidth]{Figures/chap5/irate-compression-ratio.svg}
  \caption{Comparison of Compression Ratio in different types of audio and I-Rates}
  \label{fig:irate-compression-ratio}
\end{figure}

\begin{table}[ht]
\fontsize{8}{10}\selectfont
\centering
\begin{tabularx}{\linewidth}{|m{3cm}||Y|Y||Y|Y||Y|Y||Y|Y|}
\hline
I-Rates & \multicolumn{2}{c||}{0.1} & \multicolumn{2}{c||}{1} & \multicolumn{2}{c||}{2} & \multicolumn{2}{c|}{5} \\
\hline
& Mean & SD & Mean & SD & Mean & SD & Mean & SD \\
\hline
Mixture & .898 & .025 & .875 & .034 & .873 & .035 & .871 & .036  \\
\hline
Vocals & .643 & .17 & .598 & .174 & .594 & .174 & .589 & .174 \\
\hline
Drums & .804 & .105 & .761 & .122 & .755 & .125 & .751 & .126 \\
\hline
REPET's Foreground & .834 & .042 & .78 & .053 & .774 & .054 & .768 & .055\\
\hline
REPET's Background & .88 & .027 & .855 & .037 & .852 & .038 & .85 & .039 \\
\hline
Sum & .859 & .032 & .847 & .037 & .846 & .037 & .844 & .038 \\
\hline
\end{tabularx}
\caption{Table of Compression Ratio in different types of audio and I-Rates}
\label{tab:irate-compression-ratio}
\end{table}

It can be seen in figure \ref{fig:irate-compression-ratio} and table \ref{tab:irate-compression-ratio} that all audio inputs are slightly compressed in I-Rate of 0.1 of a second and then a bit more above one second.
Most noticeably, with the increment of I-Rate, vocal audio input is compressed more than the other.
Within-subject ANOVA showed that Compression Ratio is significantly affected by the used input audio type ($F(5,245)=72.4, p<.000, \eta_{p}^{2}=.596$), the I-Rate ($F(3,147)=292, p<.000, \eta_{p}^{2}=.856$), and the interaction between them ($F(15,735)=48, p<.000, \eta_{p}^{2}=.495$).
Tukey HSD post-hoc showed that the compression is significantly worse in I-Rate of 0.1 of a second ($M=.82, SD=.12$), as compared to other used I-Rates (all $p$s $<.001$) while it showed no significant improvement of compression for the I-Rate of 1 second ($M=-.786, SD=.131$), 2 seconds ($M=.782, SD=.133$), and 5 seconds ($M=.779, SD=.134$). 
In vocals specifically, Tukey HSD showed no significant reduction of audio quality in I-Rates of 0.1 of a second, 1 second, 2 seconds and 5 seconds (all $p$s $>.319$).

In summary, the results confirmed that the Compression Ratio doesn't improve with I-Rates higher than 1 second.

\subsection{Desirability}

\begin{figure}[ht]
  \includesvg[inkscapelatex=false,  width=\linewidth]{Figures/chap5/irate-desirable.svg}
  \caption{Comparison of Desirability in different types of audio and I-Rates}
  \label{fig:irate-desirable}
\end{figure}

\begin{table}[ht]
\fontsize{8}{10}\selectfont
\centering
\begin{tabularx}{\linewidth}{|m{3cm}||Y|Y||Y|Y||Y|Y||Y|Y|}
\hline
I-Rates & \multicolumn{2}{c||}{0.1} & \multicolumn{2}{c||}{1} & \multicolumn{2}{c||}{2} & \multicolumn{2}{c|}{5} \\
\hline
& Mean & SD & Mean & SD & Mean & SD & Mean & SD \\
\hline
Mixture & .042 & .01 & .035 & .029 & .032 & .041 & .03 & .05\\
\hline
Vocals & .006 & .072 & -.205 & .368 & -.241 & .36 & -.342 & .563 \\
\hline
Drums & .012 & .068 & -.051 & .173 & -.07 & .205 & -.066 & .18 \\
\hline
REPET's Foreground & .028 & .02 & -.012 & .063 & -.022 & .065 & -.036 & .086  \\
\hline
REPET's Background & .041 & .011 & .032 & .034 & .028 & .051 & .026 & .059 \\
\hline
Sum & .043 & .011 & .036 & .033 & .032 & .045 & .031 & .051 \\
\hline
\end{tabularx}
\caption{Table of Desirability in different types of audio and I-Rates}
\label{tab:irate-desirable}
\end{table}

It can be seen in figure \ref{fig:irate-desirable} and table \ref{tab:irate-desirable} that mixture, REPET's background, and the Sum audio inputs are the most desirable.
In drums and REPET's foreground audio inputs, the increment of I-Rate makes the result slightly less desirable.
Most noticeably, in vocals, the increment of I-Rate makes the algorithm least desirable.
Within-subject ANOVA showed that audio quality is significantly affected by the used input audio type ($F(5,245)=21.8, p<.000, \eta_{p}^{2}=.308$), the I-Rate ($F(3,147)=28.5, p<.000, \eta_{p}^{2}=.368$), and the interaction between them ($F(15,735)=12.6, p<.000, \eta_{p}^{2}=.205$).
Tukey HSD post-hoc showed that the Desirability is significantly higher in I-Rate of 0.1 of a second ($M=.029, SD=.044$), as compared to other used I-Rates (all $p$s $<.001$) while it showed no significant reduction of Desirability for the I-Rate of 1 second ($M=-.028, SD=.189$), 2 seconds ($M=-.04, SD=.198$), and 5 seconds ($M=-.06, SD=.278$). In vocals specifically, Tukey HSD showed no significant reduction of Desirability between I-Rates of 1 second and 2 seconds, and I-Rates of 2 seconds and 5 seconds. However, the I-Rate of 5 seconds is significantly less desired than the I-Rate of 1 second ($p=.02$).

In summary, the results confirmed that, except for vocals, the lossy compression remains desirable with any amount of I-Rate and it stabilizes when it is higher than 1 second.

\subsection{P-Ratio}

\begin{figure}[ht]
  \includesvg[inkscapelatex=false,  width=\linewidth]{Figures/chap5/irate-pratio.svg}
  \caption{Comparison of P-Ratios in different types of audio and I-Rates}
  \label{fig:irate-pratio}
\end{figure}

\begin{table}[ht]
\fontsize{8}{10}\selectfont
\centering
\begin{tabularx}{\linewidth}{|m{3cm}||Y|Y||Y|Y||Y|Y||Y|Y|}
\hline
I-Rates & \multicolumn{2}{c||}{0.1} & \multicolumn{2}{c||}{1} & \multicolumn{2}{c||}{2} & \multicolumn{2}{c|}{5} \\
\hline
& Mean & SD & Mean & SD & Mean & SD & Mean & SD \\
\hline
Mixture & .737 & .014 & .961 & .019 & .972 & .019 & .979 & .019 \\
\hline
Vocals & .58 & .155 & .756 & .202 & .765 & .204 & .77 & .206 \\
\hline
Drums & .687 & .075 & .895 & .098 & .906 & .099 & .912 & .1 \\
\hline
REPET's Foreground & .72 & .019 & .938 & .025 & .949 & .026 & .956 & .026  \\
\hline
REPET's Background & .737 & .015 & .96 & .019 & .972 & .019 & .978 & .02 \\
\hline
Sum & .737 & .015 & .96 & .019 & .972 & .019 & .978 & .02 \\
\hline
\end{tabularx}
\caption{Table of P-Ratios in different types of audio and I-Rates}
\label{tab:irate-pratio}
\end{table}

It can be seen in figure \ref{fig:irate-pratio} and table \ref{tab:irate-pratio} that in mixture, REPET's foreground, REPET's background, and the sum audio inputs, P-Ratio starts with a lower value in 0.1 of a second, and then it increases.
In drums audio input, P-Ratio had a similar trend but was slightly lower compared to the former.
Most noticeably in vocals, P-Ratio was highly lower compared to the rest.
Within-subject ANOVA showed that P-Ratio is significantly affected by the used input audio type ($F(5,245)=40.6, p<.000, \eta_{p}^{2}=.453$), the I-Rate ($F(3,147)=20263, p<.000, \eta_{p}^{2}=.998$), and the interaction between them ($F(15,735)=40.6, p<.000, \eta_{p}^{2}=.453$).
Tukey HSD post-hoc showed that P-Ratio is significantly lower in I-Rate of 0.1 of a second ($M=.7, SD=.09$), as compared to other used I-Rates (all $p$s $<.001$) while it showed no significant improvement for the I-Rate of 1 second ($M=.912, SD=.118$), 2 seconds ($M=.923, SD=.119$), and 5 seconds ($M=.929, SD=.12$). Moreover, Tukey HSD showed that P-Ratio is significantly lower in vocals ($M=.718, SD=.207$), as compared to other used I-Rates (all $p$s $<.001$) while it showed no significant change for the mixture ($M=-.912, SD=.103$), drums ($M=.85, SD=.132$), REPET's foreground ($M=.891, SD=.102$), REPET's background ($M=.912, SD=.103$), and the sum ($M=.912, SD=.103$).

In summary, the results confirmed that, except for vocals, the lossy compression can find P-Frames with any amount of I-Rate and it stabilizes when it is higher than 1 second.

% \subsection{Reference Distance}

% \begin{figure}[ht]
%   \includesvg[inkscapelatex=false,  width=\linewidth]{Figures/chap5/irate-quantiles.svg}
%   \caption{Comparison of Reference Distances in different types of audio and I-Rates}
%   \label{fig:irate-quantiles}
% \end{figure}

% \TODO{Text needed for reference distance} 
\section{Comparison of Compression Schemas}
\label{sec:pframe-exp}

It can be concluded from section \ref{sec:i-rates} that 2 seconds is a suitable value to use for the I-Rate parameter and we only use this constant value for the P-Frame schema. Now it is time to answer the main research question: i.e "Can the described P-Frame schema improve the compression of pieces of music with repeated instrumental tones without adding perceptible differences?". To study the answer, DVs are measured as described at the beginning of chapter \ref{chapter:Exp} with an additional 3-level IV named \textbf{Schema}:

\begin{enumerate}
\item\textbf{Lossless:} The base condition which saves raw MDCT output.
\item\textbf{Only I-Frames:} The MDCT output with all I-Frames zero thresholds applied.
\item\textbf{With P-Frames:} The proposed new compression scheme.
\end{enumerate}

\noindent Following sections are reports for each measured variable.

\subsection{Audio Quality}

\begin{figure}[ht]
  \includesvg[inkscapelatex=false,  width=\linewidth]{Figures/chap5/input-peaq.svg}
  \caption[Comparison of Schemas in Audio Quality amongst different types of audio]{Comparison of Schemas in Audio Quality amongst different types of audio. Audio Quality is measured using Objective Difference Grade (ODG) in Perceptual Evaluation of Audio Quality (PEAQ) (0 = Imperceptible, -1 = Perceptible, but not annoying)}
  \label{fig:input-peaq}
\end{figure}

\begin{table}[ht]
\centering
\begin{tabularx}{\linewidth}{|X|X|X|X|X|}
\hline
Input Type & Mixture & Vocals & Drums & REPET's Foreground & REPET's Background \\
\hline
Lossless & $M=.164, SD=.009$ & $M=.151, SD=.015$ & $M=.149, SD=.03$ & $M=.15, SD=.016$ & $M=.159, SD=.009$ \\
\hline
All I-Frames & $M=.152, SD=.035$ & $M=.146, SD=.017$ & $M=.113, SD=.088$ & $M=.14, SD=.022$ & $M=.146, SD=.039$ \\
\hline
With P-Frames & $M=.114, SD=.137$ & $M=-.46, SD=.645$ & $M=-.161, SD=.479$ & $M=-.065, SD=.194$ & $M=.098, SD=.165$\\
\hline
\end{tabularx}
\caption[Table of Schemas in Audio Quality amongst different types of audio]{Table of Schemas in Audio Quality amongst different types of audio. Audio Quality is measured using Objective Difference Grade (ODG) in Perceptual Evaluation of Audio Quality (PEAQ) (0 = Imperceptible, -1 = Perceptible, but not annoying)}
\label{tab:input-peaq}
\end{table}

It can be seen in figure \ref{fig:input-peaq} and table \ref{tab:input-peaq} that lossless encoding has an ODG above zero (imperceptible) and very similar in all input types.
Similarly, all I-frame encoding is imperceptible with a slight decrease in audio quality in all input types.
However, with P-frame encoding can have an ODG below zero depending on the audio type with even more decrease in all input types.

Within-subject ANOVA showed that the audio quality is significantly affected by the type of audio  ($F(4,196)=24.4, p<.000, \eta_{p}^{2}=.332$), the used schema ($F(2,98)=56.4, p<.000, \eta_{p}^{2}=.535$), and the interaction between them ($F(8,392)=25.1, p<.000, \eta_{p}^{2}=.339$).
Tukey HSD post-hoc showed that the audio quality for vocals ($M=-.054, SD=.469$) was significantly lower than mixture ($M=.144, SD=.083$), drums ($M=.033, SD=.313$), REPET's foreground ($M=.075, SD=.15$), and REPET's background ($M=.134, SD=.1$), (all $p$s $<.007$).
Beside that, the audio quality for drums was significantly lower than the mixture, and REPET's background (all $p$s $<.001$).
It is important to mention that Tukey HSD showed no significant reduction of audio quality for having P-frames ($M=-.06, SD=.407$), all I-frames ($M=.139, SD=.049$), and lossless ($M=.155, SD=.018$).

In summary, the results confirmed that having P-Frames can significantly reduce the audio quality on vocals, and not the background sounds such as drums. However, it might be confusing why mixtures have a better sound quality with P-Frames scheme as compared to drums. This can probably be because drums audio type contains more silence and human ears can perceive differences in silence better as compared to the mixture with many accompanying instruments.

\subsection{Compression Ratio}

\begin{figure}[ht]
  \includesvg[inkscapelatex=false,  width=\linewidth]{Figures/chap5/input-compression-ratio.svg}
  \caption{Comparison of Schemas in Compression Ratio amongst different types of audio}
  \label{fig:input-compression}
\end{figure}

\begin{table}[ht]
\centering
\begin{tabularx}{\linewidth}{|X|X|X|X|X|}
\hline
Input Type & Mixture & Vocals & Drums & REPET's Foreground & REPET's Background \\
\hline
Lossless & $M=.993, SD=.019$ & $M=.717, SD=.182$ & $M=.85, SD=.123$ & $M=.91, SD=.026$ & $M=.917, SD=.021$ \\
\hline
All I-Frames & $M=.906, SD=.023$ & $M=.704, SD=.178$ & $M=.836, SD=.124$ & $M=.839, SD=.04$ & $M=.893, SD=.021$ \\
\hline
With P-Frames & $M=.873, SD=.035$ & $M=.594, SD=.174$ & $M=.755, SD=.125$ & $M=.774, SD=.054$ & $M=.893, SD=.022$ \\
\hline
\end{tabularx}
\caption{Table of Schemas in Compression Ratio amongst different types of audio}
\label{tab:input-compression}
\end{table}

It can be seen in figure \ref{fig:input-compression} and table \ref{tab:input-compression} that lossless encoding has a very similar compression ratio in the mixture, REPET's foreground, and REPET's background.
However, lossless encoding had a lower compression ratio in the vocals, and drums.
Compared to the lossless encoding, fully I-frame encoding reduced the compression ratio in all input types.
Having P-frames reduced the compression ratio even further in all input types.
Within-subject ANOVA showed that compression ratio is significantly affected by the used input audio type ($F(4,196)=51.8, p<.000, \eta_{p}^{2}=.514$), the used schema ($F(2,98)=164, p<.000, \eta_{p}^{2}=.77$), and the interaction between them ($F(8,392)=20.3, p<.000, \eta_{p}^{2}=.293$).
Tukey HSD post-hoc showed that the compression ratio with P-frame encoding ($M=-.782, SD=.133$) was significantly lower than the compression ratio for fully I-frame encoding ($M=.836, SD=.122, p<.0001$), while the fully I-frame encoding was not significantly lower than the lossless encoding ($M=.033, SD=.313, p=1$).
As for audio inputs, Tukey HSD post-hoc showed that the compression ratio for vocals ($M=.672, SD=.185$) was significantly lower than the compression ratio for mixture ($M=.904, SD=.036$), drums ($M=.814, SD=.13$), REPET's foreground ($M=.841, SD=.07$), and REPET's background ($M=.888, SD=.039$).
Beside that, the compression ratio for drums was significantly lower than the compression ratio for mixture, REPET's foreground, and REPET's background (all $p$s $<.0005$).
However, the compression ratio for drums was not significantly different than REPET's foreground ($p=154$). Similarly, the compression ratio for the mixture was not significantly different than the compression ratio for REPET's background ($p=696$).

In summary, the results confirmed that having P-Frames can significantly reduce the compression ratio.
Furthermore, the compression ratio is highly dependent on the type of input audio. Specifically, having more silence in the audio input will reduce the compression ratio.
This explains the reason that vocals and drums audio inputs have better compression ratios.

\subsection{Desirability}

\begin{figure}[ht]
  \includesvg[inkscapelatex=false,  width=\linewidth]{Figures/chap5/input-desirable.svg}
  \caption{Comparison of Compression Schemas in Desirability amongst different types of audio}
  \label{fig:input-desirable}
\end{figure}

\begin{table}[ht]
\centering
\begin{tabularx}{\linewidth}{|X|X|X|X|X|}
\hline
Input Type & Mixture & Vocals & Drums & REPET's Foreground & REPET's Background \\
\hline
Lossless & $M=.044, SD=.003$ & $M=.059, SD=.033$ & $M=.045, SD=.012$ & $M=.041, SD=.005$ & $M=.043, SD=.003$ \\
\hline
All I-Frames & $M=.042, SD=.01$ & $M=.058, SD=.03$ & $M=.034, SD=.027$ & $M=.042, SD=.007$ & $M=.041, SD=.011$ \\
\hline
With P-Frames & $M=.032, SD=.041$ & $M=-.241, SD=.36$ & $M=-.07, SD=.205$ & $M=-.022, SD=.065$ & $M=.028, SD=.051$\\
\hline
\end{tabularx}
\caption{Table of Compression Schemas in Desirability amongst different types of audio}
\label{tab:input-desirable}
\end{table}

It can be seen in figure \ref{fig:input-desirable} and table \ref{tab:input-desirable} that desirability is not much being reduced in mixture and REPET's background audio inputs with all compression schemes.
However, with P-frame encoding, desirability is slightly reduced in drums and REPET's foreground audio and it is reduced the most in vocals audio input.
Within-subject ANOVA showed that desirability is significantly affected by the used input audio type ($F(4,196)=16.3, p<.000, \eta_{p}^{2}=.25$), the used schema ($F(2,98)=46.7, p<.000, \eta_{p}^{2}=.488$), and the interaction between them ($F(8,392)=21.9, p<.000, \eta_{p}^{2}=.309$).
Tukey HSD post-hoc showed that the desirability was significantly reduced in vocals with P-frame encoding and in drums with P-frame encoding compared to all other cases (all $p$s $<.0008$). However, the desirability of having P-frames is not significantly reduced in the mixture, REPET's foreground, and REPET's background.

In summary, the results confirmed that having P-frames can be significantly undesirable in vocals. Having P-frames can also significantly reduce desirability in drums but the mean difference shows it is not as undesirable as it is for vocals.

\section{Conclusions of Experiments} 

% \TODO{Write a summary of the main conclusions of the experiments - essentially repeat in condensed form the In summary sentences of the subsections in the chapter} 

To summarize, it can be concluded from section \ref{sec:i-rates} that 2 seconds is a suitable value to use for the I-Rate parameter for the P-Frame schema. Afterwards, from section \ref{sec:pframe-exp} it can be concluded that having P-Frames can significantly reduce the Audio Quality and Desirability on vocal tracks, while the background sound is not affected as much. Furthermore, having P-Frames can have a significant improvement effect on the compression ratio by 3 to 8 percent.