\newpage
\TOCadd{Abstract}

\noindent\textbf{Supervisory Committee}
\tpbreak
\panel

\begin{center}
\textbf{ABSTRACT}
\end{center}


Music frequently 

In a Virtual Music Performance, having even 100ms latency for a group to perform a music piece together can easily make them out of sync. So such performances benefit from domain-specific audio compressions. One possible approach to improve audio compression is to leverage the repeating aspect of musical beats, vamps, and rhythms. In this thesis, after providing some background, a simple implementation of perceptual audio compression is described that was used for the exploration of ideas. Then, a new audio compression scheme is described and tested with an experimental methodology over a publicly available dataset. Results show promising results in compression ratio and audio quality on rhythmic mixtures and background music. Lastly, some ideas are suggested for future works.
