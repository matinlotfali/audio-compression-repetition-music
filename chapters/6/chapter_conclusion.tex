\startchapter{Conclusions and Future Work}
\label{chapter:concl}

Virtual Music Performances can benefit from domain-specific audio compressions. One approach to improve audio compression is to leverage the repeating aspect of musical beats, vamps, and rhythms. In this thesis, after providing some background, a simple implementation of perceptual audio compression is described that was used for the exploration of ideas. Then, inspired by the MPEG-1 video encoder, the idea of using I-Frame and P-Frames over the MDCT of audio signals is described, implemented, and tested in one example. And lastly, an experimental methodology is designed using the MUSDB-HQ dataset and results were reported with descriptive and inferential statistical analysis.

The proposed audio compression scheme shows promising results on mixture and background music in which musical rhythms can be found but for a vocal track, it is rather harmful. Even though these results were found before I read about Cunningham et al. at the time of writing this thesis, their findings confirm the experimental results of this thesis. However, due to lack of time, this new idea was not compared with actual MP3 and AAC encoders but rather it was compared to a simplified implementation while all other lossy techniques were disabled. Additionally, because the idea was implemented with the Python programming language, the processing time was not analyzed or compared. Therefore, for future work, I suggest that the proposed method is implemented using the C++ programming language, all lossy techniques for perceptual audio compression are applied, and then it would be compared with both MP3 and AAC encoders in audio quality, compression ratio, and latency. This can allow us to see if the idea is beneficial for a real NMP scenario.

Another direction for  future work stems from the finding that P-Frames are mostly referenced to their closes first or second frames (see figure \ref{fig:hist}). This probably means there is more similarity between frames that contain the continuous or decaying signal of the same tone as compared to another tone that is being played at a different time. So I suggest the P-Frame be modified in favour of this decaying signal. One possible modification can be to save the ratio of the absolute sum of two frames ($r = |F_2|/|F_1|$) and use the ratio when calculating the difference between two frames ($P_2=F_2-rF_1$). This might cause the P-Frame to contain more zero elements after the threshold. The list of all ratios are needed to be stored in the encoded stream alongside the list of references, but overall I hypothesize that it would improve the compression ratio of the encoded bitstream.



% My first rule for this chapter is to avoid finishing it with a section talking about future work. It may seem logical, yet it also appears to give a list of all items which remain undone! It is not the best way psychologically.

% This chapter should contain a mirror of the introduction, where a summary of the \textit{extraordinary} new results and their wonderful attributes should be stated first, followed by an executive summary of how this new solution was arrived at. Consider the practical fact that this chapter will be read quickly at the beginning of a review (thus it needs to provide a strong impact) and then again in depth at the very end, perhaps a few days after the details of the previous 3 chapters have been somehow forgotten. Reinforcement of the positive is the key strategy here, without of course blowing hot air.

% One other consideration is that some people like to join the chapter containing the analysis with the only with conclusions. This can indeed work very well in certain topics.

% Finally, the conclusions do not appear only in this chapter. This sample mini thesis lacks a feature which I regard as absolutely necessary, namely a short paragraph at the end of each chapter giving a brief summary of what was presented together with a one sentence preview as to what might expect the connection to be with the next chapter(s). You are writing a story, the \textit{story of your wonderful research work}. A story needs a line connecting all its parts and you are responsible for these linkages.
