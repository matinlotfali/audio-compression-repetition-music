\section{Effects of I-Rate on the Compression Scheme}
\label{sec:i-rates}

As described in section \ref{sec:idea}, the I-Rate parameter is the rate of forced I-Frames that P-Frames are not allowed to refer to any frame before them. A larger number in the I-Rate means P-Frames are allowed to look more back in time to find a suitable referenced frame to reuse. Additionally, it means more comparisons and more CPU time. In this section, I try to answer the question: What is the suitable amount of time for P-Frames to look back? My hypothesis is that, in the worst case, beats are repeated once per second. Therefore, 2 seconds for the I-Rate should be sufficient and there is no point in going back more in time to find a reference.

To test this hypothesis, DVs are measured as described at the beginning of chapter \ref{chapter:Exp} with an additional 4-level IV named I-Rates: 0.1 of a second, 1 second, 2 seconds, and 5 seconds. The following sections are reports for each measured variable.

\subsection{Audio Quality}

\begin{figure}[ht]
  \includesvg[inkscapelatex=false,  width=\linewidth]{Figures/chap5/irate-peaq.svg}
  \caption[Comparison of Audio Quality in different types of audio and I-Rates.]{Comparison of Audio Quality in different types of audio and I-Rates. Audio Quality is measured using Objective Difference Grade (ODG) in Perceptual Evaluation of Audio Quality (PEAQ) (0 = Imperceptible, -1 = Perceptible, but not annoying)}
  \label{fig:irate-peaq}
\end{figure}

\begin{table}[ht]
\fontsize{8}{10}\selectfont
\centering
\begin{tabularx}{\linewidth}{|m{3cm}||Y|Y||Y|Y||Y|Y||Y|Y|}
\hline
I-Rates & \multicolumn{2}{c||}{0.1} & \multicolumn{2}{c||}{1} & \multicolumn{2}{c||}{2} & \multicolumn{2}{c|}{5} \\
\hline
& Mean & SD & Mean & SD & Mean & SD & Mean & SD \\
\hline
Mixture & .15 & .037 & .125 & .099 & .114 & .137 & .105 & .166 \\
\hline
Vocals & .014 & .18 & -.349 & .513 & -.46 & .645 & -.603 & .805 \\
\hline
Drums & .051 & .187 & -.111 & .41 & -.161 & .479 & -.162 & .464 \\
\hline
REPET's Foreground & .092 & .064 & -.036 & .185 & -.065 & .194 & -.107 & .25 \\
\hline
REPET's Background & .143 & .039 & .112 & .098 & .098 & .165 & .089 & .187 \\
\hline
Sum & .148 & .038 & .121 & .107 & .111 & .148 & .104 & .167 \\
\hline
\end{tabularx}
\caption[Table of Audio Quality in different types of audio and I-Rates.]{Table of Audio Quality in different types of audio and I-Rates. Audio Quality is measured using Objective Difference Grade (ODG) in Perceptual Evaluation of Audio Quality (PEAQ) (0 = Imperceptible, -1 = Perceptible, but not annoying)}
\label{tab:irate-peaq}
\end{table}

It can be seen in figure \ref{fig:irate-peaq} and table \ref{tab:irate-peaq} that mixture, REPET's background, and the Sum audio inputs are imperceptible in all I-Rates.
In drums and REPET's foreground, however, with the increment of I-Rate, the audio quality is slightly reduced.
Most noticeably, with the increment of I-Rate, the audio quality in vocals are highly reduced.
Within-subject ANOVA showed that the audio quality is significantly affected by the used input audio type ($F(5,245)=31.4, p<.000, \eta_{p}^{2}=.39$), the I-Rate ($F(3,147)=44.8, p<.000, \eta_{p}^{2}=.477$), and the interaction between them ($F(15,735)=18.7, p<.000, \eta_{p}^{2}=.276$).
Tukey HSD post-hoc showed that the audio quality is significantly higher in I-Rate of 0.1 of a second ($M=.1, SD=.122$), as compared to other used I-Rates (all $p$s $<.001$) while it showed no significant reduction of audio quality for the I-Rate of 1 second ($M=-.023, SD=.333$), 2 seconds ($M=-.06, SD=.407$), and 5 seconds ($M=-.096, SD=.479$). In vocals specifically, Tukey HSD showed no significant reduction of audio quality between I-Rates of 1 second and 2 seconds, and I-Rates of 2 seconds and 5 seconds. However, the audio quality between I-Rates of 1 second and 5 seconds are significantly lowered ($p=.011$).

In summary, the results confirmed that, except for vocals, the lossy compression remains imperceptible with any amount of I-Rate and it stabilizes when it is higher than 1 second.

\subsection{Compression Ratio}

\begin{figure}[ht]
  \includesvg[inkscapelatex=false,  width=\linewidth]{Figures/chap5/irate-compression-ratio.svg}
  \caption{Comparison of Compression Ratio in different types of audio and I-Rates}
  \label{fig:irate-compression-ratio}
\end{figure}

\begin{table}[ht]
\fontsize{8}{10}\selectfont
\centering
\begin{tabularx}{\linewidth}{|m{3cm}||Y|Y||Y|Y||Y|Y||Y|Y|}
\hline
I-Rates & \multicolumn{2}{c||}{0.1} & \multicolumn{2}{c||}{1} & \multicolumn{2}{c||}{2} & \multicolumn{2}{c|}{5} \\
\hline
& Mean & SD & Mean & SD & Mean & SD & Mean & SD \\
\hline
Mixture & .898 & .025 & .875 & .034 & .873 & .035 & .871 & .036  \\
\hline
Vocals & .643 & .17 & .598 & .174 & .594 & .174 & .589 & .174 \\
\hline
Drums & .804 & .105 & .761 & .122 & .755 & .125 & .751 & .126 \\
\hline
REPET's Foreground & .834 & .042 & .78 & .053 & .774 & .054 & .768 & .055\\
\hline
REPET's Background & .88 & .027 & .855 & .037 & .852 & .038 & .85 & .039 \\
\hline
Sum & .859 & .032 & .847 & .037 & .846 & .037 & .844 & .038 \\
\hline
\end{tabularx}
\caption{Table of Compression Ratio in different types of audio and I-Rates}
\label{tab:irate-compression-ratio}
\end{table}

It can be seen in figure \ref{fig:irate-compression-ratio} and table \ref{tab:irate-compression-ratio} that all audio inputs are slightly compressed in I-Rate of 0.1 of a second and then a bit more above one second.
Most noticeably, with the increment of I-Rate, vocal audio input is compressed more than the other.
Within-subject ANOVA showed that Compression Ratio is significantly affected by the used input audio type ($F(5,245)=72.4, p<.000, \eta_{p}^{2}=.596$), the I-Rate ($F(3,147)=292, p<.000, \eta_{p}^{2}=.856$), and the interaction between them ($F(15,735)=48, p<.000, \eta_{p}^{2}=.495$).
Tukey HSD post-hoc showed that the compression is significantly worse in I-Rate of 0.1 of a second ($M=.82, SD=.12$), as compared to other used I-Rates (all $p$s $<.001$) while it showed no significant improvement of compression for the I-Rate of 1 second ($M=-.786, SD=.131$), 2 seconds ($M=.782, SD=.133$), and 5 seconds ($M=.779, SD=.134$). 
In vocals specifically, Tukey HSD showed no significant reduction of audio quality in I-Rates of 0.1 of a second, 1 second, 2 seconds and 5 seconds (all $p$s $>.319$).

In summary, the results confirmed that the Compression Ratio doesn't improve with I-Rates higher than 1 second.

\subsection{Desirability}

\begin{figure}[ht]
  \includesvg[inkscapelatex=false,  width=\linewidth]{Figures/chap5/irate-desirable.svg}
  \caption{Comparison of Desirability in different types of audio and I-Rates}
  \label{fig:irate-desirable}
\end{figure}

\begin{table}[ht]
\fontsize{8}{10}\selectfont
\centering
\begin{tabularx}{\linewidth}{|m{3cm}||Y|Y||Y|Y||Y|Y||Y|Y|}
\hline
I-Rates & \multicolumn{2}{c||}{0.1} & \multicolumn{2}{c||}{1} & \multicolumn{2}{c||}{2} & \multicolumn{2}{c|}{5} \\
\hline
& Mean & SD & Mean & SD & Mean & SD & Mean & SD \\
\hline
Mixture & .042 & .01 & .035 & .029 & .032 & .041 & .03 & .05\\
\hline
Vocals & .006 & .072 & -.205 & .368 & -.241 & .36 & -.342 & .563 \\
\hline
Drums & .012 & .068 & -.051 & .173 & -.07 & .205 & -.066 & .18 \\
\hline
REPET's Foreground & .028 & .02 & -.012 & .063 & -.022 & .065 & -.036 & .086  \\
\hline
REPET's Background & .041 & .011 & .032 & .034 & .028 & .051 & .026 & .059 \\
\hline
Sum & .043 & .011 & .036 & .033 & .032 & .045 & .031 & .051 \\
\hline
\end{tabularx}
\caption{Table of Desirability in different types of audio and I-Rates}
\label{tab:irate-desirable}
\end{table}

It can be seen in figure \ref{fig:irate-desirable} and table \ref{tab:irate-desirable} that mixture, REPET's background, and the Sum audio inputs are the most desirable.
In drums and REPET's foreground audio inputs, the increment of I-Rate makes the result slightly less desirable.
Most noticeably, in vocals, the increment of I-Rate makes the algorithm least desirable.
Within-subject ANOVA showed that audio quality is significantly affected by the used input audio type ($F(5,245)=21.8, p<.000, \eta_{p}^{2}=.308$), the I-Rate ($F(3,147)=28.5, p<.000, \eta_{p}^{2}=.368$), and the interaction between them ($F(15,735)=12.6, p<.000, \eta_{p}^{2}=.205$).
Tukey HSD post-hoc showed that the Desirability is significantly higher in I-Rate of 0.1 of a second ($M=.029, SD=.044$), as compared to other used I-Rates (all $p$s $<.001$) while it showed no significant reduction of Desirability for the I-Rate of 1 second ($M=-.028, SD=.189$), 2 seconds ($M=-.04, SD=.198$), and 5 seconds ($M=-.06, SD=.278$). In vocals specifically, Tukey HSD showed no significant reduction of Desirability between I-Rates of 1 second and 2 seconds, and I-Rates of 2 seconds and 5 seconds. However, the I-Rate of 5 seconds is significantly less desired than the I-Rate of 1 second ($p=.02$).

In summary, the results confirmed that, except for vocals, the lossy compression remains desirable with any amount of I-Rate and it stabilizes when it is higher than 1 second.

\subsection{P-Ratio}

\begin{figure}[ht]
  \includesvg[inkscapelatex=false,  width=\linewidth]{Figures/chap5/irate-pratio.svg}
  \caption{Comparison of P-Ratios in different types of audio and I-Rates}
  \label{fig:irate-pratio}
\end{figure}

\begin{table}[ht]
\fontsize{8}{10}\selectfont
\centering
\begin{tabularx}{\linewidth}{|m{3cm}||Y|Y||Y|Y||Y|Y||Y|Y|}
\hline
I-Rates & \multicolumn{2}{c||}{0.1} & \multicolumn{2}{c||}{1} & \multicolumn{2}{c||}{2} & \multicolumn{2}{c|}{5} \\
\hline
& Mean & SD & Mean & SD & Mean & SD & Mean & SD \\
\hline
Mixture & .737 & .014 & .961 & .019 & .972 & .019 & .979 & .019 \\
\hline
Vocals & .58 & .155 & .756 & .202 & .765 & .204 & .77 & .206 \\
\hline
Drums & .687 & .075 & .895 & .098 & .906 & .099 & .912 & .1 \\
\hline
REPET's Foreground & .72 & .019 & .938 & .025 & .949 & .026 & .956 & .026  \\
\hline
REPET's Background & .737 & .015 & .96 & .019 & .972 & .019 & .978 & .02 \\
\hline
Sum & .737 & .015 & .96 & .019 & .972 & .019 & .978 & .02 \\
\hline
\end{tabularx}
\caption{Table of P-Ratios in different types of audio and I-Rates}
\label{tab:irate-pratio}
\end{table}

It can be seen in figure \ref{fig:irate-pratio} and table \ref{tab:irate-pratio} that in mixture, REPET's foreground, REPET's background, and the sum audio inputs, P-Ratio starts with a lower value in 0.1 of a second, and then it increases.
In drums audio input, P-Ratio had a similar trend but was slightly lower compared to the former.
Most noticeably in vocals, P-Ratio was highly lower compared to the rest.
Within-subject ANOVA showed that P-Ratio is significantly affected by the used input audio type ($F(5,245)=40.6, p<.000, \eta_{p}^{2}=.453$), the I-Rate ($F(3,147)=20263, p<.000, \eta_{p}^{2}=.998$), and the interaction between them ($F(15,735)=40.6, p<.000, \eta_{p}^{2}=.453$).
Tukey HSD post-hoc showed that P-Ratio is significantly lower in I-Rate of 0.1 of a second ($M=.7, SD=.09$), as compared to other used I-Rates (all $p$s $<.001$) while it showed no significant improvement for the I-Rate of 1 second ($M=.912, SD=.118$), 2 seconds ($M=.923, SD=.119$), and 5 seconds ($M=.929, SD=.12$). Moreover, Tukey HSD showed that P-Ratio is significantly lower in vocals ($M=.718, SD=.207$), as compared to other used I-Rates (all $p$s $<.001$) while it showed no significant change for the mixture ($M=-.912, SD=.103$), drums ($M=.85, SD=.132$), REPET's foreground ($M=.891, SD=.102$), REPET's background ($M=.912, SD=.103$), and the sum ($M=.912, SD=.103$).

In summary, the results confirmed that, except for vocals, the lossy compression can find P-Frames with any amount of I-Rate and it stabilizes when it is higher than 1 second.

% \subsection{Reference Distance}

% \begin{figure}[ht]
%   \includesvg[inkscapelatex=false,  width=\linewidth]{Figures/chap5/irate-quantiles.svg}
%   \caption{Comparison of Reference Distances in different types of audio and I-Rates}
%   \label{fig:irate-quantiles}
% \end{figure}

% \TODO{Text needed for reference distance} 