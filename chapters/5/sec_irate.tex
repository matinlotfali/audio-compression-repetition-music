\section{Effects of I-Rate on the Compression Scheme}
\label{sec:i-rates}

As described in section \ref{sec:idea}, the I-Rate parameter is the rate of forced I-Frames that P-Frames are not allowed to refer to any frame before them. A larger number in the I-Rate means P-Frames are allowed to look more back in time to find a suitable referenced frame to reuse. Additionally, it means more comparisons and more CPU time. In this section, I try to answer the question: What is the suitable amount of time for P-Frames to look back? My hypothesis is that, in the worst case, beats are repeated once per second. Therefore, 2 seconds for the I-Rate should be sufficient and there is no point in going back more in time to find a reference.

To test this hypothesis, DVs are measured as described at the beginning of chapter \ref{chapter:Exp} with an additional 4-level IV named I-Rates: 0.1 of a second, 1 second, 2 seconds, and 5 seconds. The following sections are reports for each measured variable.

\subsection{Audio Quality}

\begin{figure}[ht]
  \includesvg[inkscapelatex=false,  width=\linewidth]{Figures/chap5/irate-peaq.svg}
  \caption[Comparison of Audio Quality in different types of audio and I-Rates.]{Comparison of Audio Quality in different types of audio and I-Rates. Audio Quality is measured using Objective Difference Grade (ODG) in Perceptual Evaluation of Audio Quality (PEAQ) (0 = Imperceptible, -1 = Perceptible, but not annoying)}
  \label{fig:irate-peaq}
\end{figure}

\begin{table}[ht]
\centering
\begin{tabularx}{\linewidth}{|X|X|X|X|X|}
\hline
I-Rates & 0.1 & 1 & 2 & 5 \\
\hline
Mixture & $M=.15, SD=.037$ & $M=.125, SD=.099$ & $M=.114, SD=.137$ & $M=.105, SD=.166$ \\
\hline
Vocals & $M=.014, SD=.18$ & $M=-.349, SD=.513$ & $M=-.46, SD=.645$ & $M=-.603, SD=.805$ \\
\hline
Drums & $M=.051, SD=.187$ & $M=-.111, SD=.41$ & $M=-.161, SD=.479$ & $M=-.162, SD=.464$ \\
\hline
REPET's Foreground & $M=.092, SD=.064$ & $M=-.036, SD=.185$ & $M=-.065, SD=.194$ & $M=-.107, SD=.25$ \\
\hline
REPET's Background & $M=.143, SD=.039$ & $M=.112, SD=.098$ & $M=.098, SD=.165$ & $M=.089, SD=.187$ \\
\hline
Sum & $M=.148, SD=.038$ & $M=.121, SD=.107$ & $M=.111, SD=.148$ & $M=.104, SD=.167$ \\
\hline
\end{tabularx}
\caption[Table of Audio Quality in different types of audio and I-Rates.]{Table of Audio Quality in different types of audio and I-Rates. Audio Quality is measured using Objective Difference Grade (ODG) in Perceptual Evaluation of Audio Quality (PEAQ) (0 = Imperceptible, -1 = Perceptible, but not annoying)}
\label{tab:irate-peaq}
\end{table}

It can be seen in figure \ref{fig:irate-peaq} and table \ref{tab:irate-peaq} that mixture, REPET's background, and the Sum audio inputs are imperceptible in all I-Rates.
In drums and REPET's foreground, however, with the increment of I-Rate, the audio quality is slightly reduced.
Most noticeably, with the increment of I-Rate, the audio quality in vocals are highly reduced.
Within-subject ANOVA showed that the audio quality is significantly affected by the used input audio type ($F(5,245)=31.4, p<.000, \eta_{p}^{2}=.39$), the I-Rate ($F(3,147)=44.8, p<.000, \eta_{p}^{2}=.477$), and the interaction between them ($F(15,735)=18.7, p<.000, \eta_{p}^{2}=.276$).
Tukey HSD post-hoc showed that the audio quality is significantly higher in I-Rate of 0.1 of a second ($M=.1, SD=.122$), as compared to other used I-Rates (all $p$s $<.001$) while it showed no significant reduction of audio quality for the I-Rate of 1 second ($M=-.023, SD=.333$), 2 seconds ($M=-.06, SD=.407$), and 5 seconds ($M=-.096, SD=.479$). In vocals specifically, Tukey HSD showed no significant reduction of audio quality between I-Rates of 1 second and 2 seconds, and I-Rates of 2 seconds and 5 seconds. However, the audio quality between I-Rates of 1 second and 5 seconds are significantly lowered ($p=.011$).

In summary, the results confirmed that, except for vocals, the lossy compression remains imperceptible with any amount of I-Rate and it stabilizes when it is higher than 1 second.

\subsection{Compression Ratio}

\begin{figure}[ht]
  \includesvg[inkscapelatex=false,  width=\linewidth]{Figures/chap5/irate-compression-ratio.svg}
  \caption{Comparison of Compression Ratio in different types of audio and I-Rates}
  \label{fig:irate-compression-ratio}
\end{figure}

\begin{table}[ht]
\centering
\begin{tabularx}{\linewidth}{|X|X|X|X|X|}
\hline
I-Rates & 0.1 & 1 & 2 & 5 \\
\hline
Mixture & $M=.898, SD=.025$ & $M=.875, SD=.034$ & $M=.873, SD=.035$ & $M=.871, SD=.036$  \\
\hline
Vocals & $M=.643, SD=.17$ & $M=.598, SD=.174$ & $M=.594, SD=.174$ & $M=.589, SD=.174$ \\
\hline
Drums & $M=.804, SD=.105$ & $M=.761, SD=.122$ & $M=.755, SD=.125$ & $M=.751, SD=.126$ \\
\hline
REPET's Foreground & $M=.834, SD=.042$ & $M=.78, SD=.053$ & $M=.774, SD=.054$ & $M=.768, SD=.055$\\
\hline
REPET's Background & $M=.88, SD=.027$ & $M=.855, SD=.037$ & $M=.852, SD=.038$ & $M=.85, SD=.039$ \\
\hline
Sum & $M=.859, SD=.032$ & $M=.847, SD=.037$ & $M=.846, SD=.037$ & $M=.844, SD=.038$ \\
\hline
\end{tabularx}
\caption{Table of Compression Ratio in different types of audio and I-Rates}
\label{tab:irate-compression-ratio}
\end{table}

It can be seen in figure \ref{fig:irate-compression-ratio} and table \ref{tab:irate-compression-ratio} that all audio inputs are slightly compressed in I-Rate of 0.1 of a second and then a bit more above one second.
Most noticeably, with the increment of I-Rate, vocal audio input is compressed more than the other.
Within-subject ANOVA showed that Compression Ratio is significantly affected by the used input audio type ($F(5,245)=72.4, p<.000, \eta_{p}^{2}=.596$), the I-Rate ($F(3,147)=292, p<.000, \eta_{p}^{2}=.856$), and the interaction between them ($F(15,735)=48, p<.000, \eta_{p}^{2}=.495$).
Tukey HSD post-hoc showed that the compression is significantly worse in I-Rate of 0.1 of a second ($M=.82, SD=.12$), as compared to other used I-Rates (all $p$s $<.001$) while it showed no significant improvement of compression for the I-Rate of 1 second ($M=-.786, SD=.131$), 2 seconds ($M=.782, SD=.133$), and 5 seconds ($M=.779, SD=.134$). 
In vocals specifically, Tukey HSD showed no significant reduction of audio quality in I-Rates of 0.1 of a second, 1 second, 2 seconds and 5 seconds (all $p$s $>.319$).

In summary, the results confirmed that the Compression Ratio doesn't improve with I-Rates higher than 1 second.

\subsection{Desirability}

\begin{figure}[ht]
  \includesvg[inkscapelatex=false,  width=\linewidth]{Figures/chap5/irate-desirable.svg}
  \caption{Comparison of Desirability in different types of audio and I-Rates}
  \label{fig:irate-desirable}
\end{figure}

\begin{table}[ht]
\centering
\begin{tabularx}{\linewidth}{|X|X|X|X|X|}
\hline
I-Rates & 0.1 & 1 & 2 & 5 \\
\hline
Mixture & $M=.042, SD=.01$ & $M=.035, SD=.029$ & $M=.032, SD=.041$ & $M=.03, SD=.05$\\
\hline
Vocals & $M=.006, SD=.072$ & $M=-.205, SD=.368$ & $M=-.241, SD=.36$ & $M=-.342, SD=.563$ \\
\hline
Drums & $M=.012, SD=.068$ & $M=-.051, SD=.173$ & $M=-.07, SD=.205$ & $M=-.066, SD=.18$ \\
\hline
REPET's Foreground & $M=.028, SD=.02$ & $M=-.012, SD=.063$ & $M=-.022, SD=.065$ & $M=-.036, SD=.086$  \\
\hline
REPET's Background & $M=.041, SD=.011$ & $M=.032, SD=.034$ & $M=.028, SD=.051$ & $M=.026, SD=.059$ \\
\hline
Sum & $M=.043, SD=.011$ & $M=.036, SD=.033$ & $M=.032, SD=.045$ & $M=.031, SD=.051$ \\
\hline
\end{tabularx}
\caption{Table of Desirability in different types of audio and I-Rates}
\label{tab:irate-desirable}
\end{table}

It can be seen in figure \ref{fig:irate-desirable} and table \ref{tab:irate-desirable} that mixture, REPET's background, and the Sum audio inputs are the most desirable.
In drums and REPET's foreground audio inputs, the increment of I-Rate makes the result slightly less desirable.
Most noticeably, in vocals, the increment of I-Rate makes the algorithm least desirable.
Within-subject ANOVA showed that audio quality is significantly affected by the used input audio type ($F(5,245)=21.8, p<.000, \eta_{p}^{2}=.308$), the I-Rate ($F(3,147)=28.5, p<.000, \eta_{p}^{2}=.368$), and the interaction between them ($F(15,735)=12.6, p<.000, \eta_{p}^{2}=.205$).
Tukey HSD post-hoc showed that the Desirability is significantly higher in I-Rate of 0.1 of a second ($M=.029, SD=.044$), as compared to other used I-Rates (all $p$s $<.001$) while it showed no significant reduction of Desirability for the I-Rate of 1 second ($M=-.028, SD=.189$), 2 seconds ($M=-.04, SD=.198$), and 5 seconds ($M=-.06, SD=.278$). In vocals specifically, Tukey HSD showed no significant reduction of Desirability between I-Rates of 1 second and 2 seconds, and I-Rates of 2 seconds and 5 seconds. However, the I-Rate of 5 seconds is significantly less desired than the I-Rate of 1 second ($p=.02$).

In summary, the results confirmed that, except for vocals, the lossy compression remains desirable with any amount of I-Rate and it stabilizes when it is higher than 1 second.

\subsection{P-Ratio}

\begin{figure}[ht]
  \includesvg[inkscapelatex=false,  width=\linewidth]{Figures/chap5/irate-pratio.svg}
  \caption{Comparison of P-Ratios in different types of audio and I-Rates}
  \label{fig:irate-pratio}
\end{figure}

\begin{table}[ht]
\centering
\begin{tabularx}{\linewidth}{|X|X|X|X|X|}
\hline
I-Rates & 0.1 & 1 & 2 & 5 \\
\hline
Mixture & $M=.737, SD=.014$ & $M=.961, SD=.019$ & $M=.972, SD=.019$ & $M=.979, SD=.019$ \\
\hline
Vocals & $M=.58, SD=.155$ & $M=.756, SD=.202$ & $M=.765, SD=.204$ & $M=.77, SD=.206$ \\
\hline
Drums & $M=.687, SD=.075$ & $M=.895, SD=.098$ & $M=.906, SD=.099$ & $M=.912, SD=.1$ \\
\hline
REPET's Foreground & $M=.72, SD=.019$ & $M=.938, SD=.025$ & $M=.949, SD=.026$ & $M=.956, SD=.026$  \\
\hline
REPET's Background & $M=.737, SD=.015$ & $M=.96, SD=.019$ & $M=.972, SD=.019$ & $M=.978, SD=.02$ \\
\hline
Sum & $M=.737, SD=.015$ & $M=.96, SD=.019$ & $M=.972, SD=.019$ & $M=.978, SD=.02$ \\
\hline
\end{tabularx}
\caption{Table of P-Ratios in different types of audio and I-Rates}
\label{tab:irate-pratio}
\end{table}

It can be seen in figure \ref{fig:irate-pratio} and table \ref{tab:irate-pratio} that in mixture, REPET's foreground, REPET's background, and the sum audio inputs, P-Ratio starts with a lower value in 0.1 of a second, and then it increases.
In drums audio input, P-Ratio had a similar trend but was slightly lower compared to the former.
Most noticeably in vocals, P-Ratio was highly lower compared to the rest.
Within-subject ANOVA showed that P-Ratio is significantly affected by the used input audio type ($F(5,245)=40.6, p<.000, \eta_{p}^{2}=.453$), the I-Rate ($F(3,147)=20263, p<.000, \eta_{p}^{2}=.998$), and the interaction between them ($F(15,735)=40.6, p<.000, \eta_{p}^{2}=.453$).
Tukey HSD post-hoc showed that P-Ratio is significantly lower in I-Rate of 0.1 of a second ($M=.7, SD=.09$), as compared to other used I-Rates (all $p$s $<.001$) while it showed no significant improvement for the I-Rate of 1 second ($M=.912, SD=.118$), 2 seconds ($M=.923, SD=.119$), and 5 seconds ($M=.929, SD=.12$). Moreover, Tukey HSD showed that P-Ratio is significantly lower in vocals ($M=.718, SD=.207$), as compared to other used I-Rates (all $p$s $<.001$) while it showed no significant change for the mixture ($M=-.912, SD=.103$), drums ($M=.85, SD=.132$), REPET's foreground ($M=.891, SD=.102$), REPET's background ($M=.912, SD=.103$), and the sum ($M=.912, SD=.103$).

In summary, the results confirmed that, except for vocals, the lossy compression can find P-Frames with any amount of I-Rate and it stabilizes when it is higher than 1 second.

% \subsection{Reference Distance}

% \begin{figure}[ht]
%   \includesvg[inkscapelatex=false,  width=\linewidth]{Figures/chap5/irate-quantiles.svg}
%   \caption{Comparison of Reference Distances in different types of audio and I-Rates}
%   \label{fig:irate-quantiles}
% \end{figure}

% \TODO{Text needed for reference distance} 